% This is samplepaper.tex, a sample chapter demonstrating the
% LLNCS macro package for Springer Computer Science proceedings;
% Version 2.20 of 2017/10/04
%
\documentclass[runningheads]{llncs}
%
\usepackage{graphicx}
\usepackage{url}
% \usepackage[scale=0.69]{geometry}
% Used for displaying a sample figure. If possible, figure files should
% be included in EPS format.
%
% If you use the hyperref package, please uncomment the following line
% to display URLs in blue roman font according to Springer's eBook style:
% \renewcommand\UrlFont{\color{blue}\rmfamily}

\newcommand{\Jupyter}{\textbf{Jupyter}}

\begin{document}
%
\title{Proposition de poster : utiliser des notebooks \Jupyter{} pour l'enseignement du langage OCaml}%\thanks{Supported by ENS Rennes and CentraleSup{\'e}lec.}}
%
\titlerunning{Proposition de poster : des notebooks \Jupyter{} pour enseigner OCaml}
% % If the paper title is too long for the running head, you can set
% % an abbreviated paper title here
% %
% \author{Lilian Besson\inst{1}\orcidID{0000-0003-2767-2563}}
% %
% \authorrunning{Lilian Besson}
% % First names are abbreviated in the running head.
% % If there are more than two authors, 'et al.' is used.
% %
% \institute{{\'E}cole Normale Sup{\'e}rieure de Rennes, Bruz, France\\
% \email{lilian.besson{@}ens-rennes.fr}\\
% \url{http://www.ens-rennes.fr/}}
%
\maketitle              % typeset the header of the contribution
%
\begin{abstract}
    % The abstract should briefly summarize the contents of the paper in 150--250 words.
    % https://www.didapro.org/8/contributions-dates/
    % I propose a poster which will present the \Jupyter{} notebooks, and give a good overview of the \Jupyter{} ecosystem, focussing on its use for the OCaml language \cite{ocaml}.
    % OCaml is the language used for teaching theoretical computer science in prep' schools in France (CPGE), and using \Jupyter{} notebooks is nowadays the best solution for interactive and modern interface to present code to your students.
    % %
    % In a dense but precise A5 poster, I can explain how one can easily use \Jupyter{} notebooks with OCaml.
    % Notebooks can be used for lecture material, to obtain one document that contains text, maths, figures, code snippets and the outputs of their execution.
    % \Jupyter{} Notebooks are also an excellent tool to use for practical sessions in studying theoretical computer science using OCaml as the illustration language.

    Je propose un poster qui présentera les notebooks \Jupyter{}, et donne un bon aperçu de l'environnement \Jupyter, se concentrant sur son utilisation pour le langage OCaml \cite{ocaml}.
    OCaml est le langage utilisé pour l'enseignement de l'informatique théorique dans les classes préparatoires en France (CPGE), et l'utilisation des notebooks \Jupyter{} est aujourd'hui la meilleure solution d'interface interactive et moderne pour présenter du code à vos élèves.

    % Dans un poster A5 dense mais précis, je peux expliquer comment on peut facilement écrire des notebooks interactifs \Jupyter{} pour le langage de programmation Python.
    % Les notebooks peuvent être utilisés pour du matériel de cours, pour obtenir un document qui contient du texte, des maths, des figures, des extraits de code et les résultats de leurs exécutions.
    % Les notebooks \Jupyter{} sont également un excellent outil pour les travaux pratiques d'informatique, mais aussi pour toute autre science avec des calculs numériques ou l'analyse de données (mathématiques, physique et chimie, ingénierie, etc), et comme modèle ou squelette pouvant être remis aux étudiants, et collectés et évalués après la session pratique.
    %
    Dans un poster A5 dense mais précis, je peux expliquer comment on peut facilement utiliser \Jupyter{} avec OCaml.
    Les notebooks peuvent être utilisés pour du matériel de cours, pour obtenir un document qui contient du texte, des maths, des figures, des extraits de code et les résultats de leurs exécutions.
    Les notebooks \Jupyter{} sont également un excellent outil à utiliser pour des sessions pratiques dans l'étude de l'informatique théorique en utilisant OCaml comme langage d'illustration.

\keywords{Jupyter \and Notebook \and OCaml \and Outil open-source \and Enseignement de la programmation \and Classes préparatoires \and CPGE \and Enseignement de l'informatique théorique.}
\end{abstract}
%
%
%

% I will focus on using notebooks with the OCaml language and for tutorials and practical sessions at the prep' schools (CPGE) level in France.
% The default IDE that is still used in many class in CPGE in France to teach and practice with OCaml is the Tuareg mode of Emacs, which is the complete opposite of a modern IDE.
% Moreover, \Jupyter{} notebooks with OCaml, and a recent initiative\footnote{See \url{github.com/louisabraham/domical}} allow anybody to interact with documents interlacing text, maths, figures and OCaml code cells, directly from a web browser and without installing any software, which is a clear advantage as OCaml is not usually installed by default on modern computers (contrarily to Python which is shipped on any GNU/Linux and Mac OS X computers nowadays).

Je me concentrerai sur l'utilisation de notebooks avec le langage OCaml, pour des tutoriels et des sessions pratiques au niveau des écoles préparatoires (CPGE) en France.
L'IDE par défaut qui est encore utilisé dans de nombreuses classes du CPGE en France pour enseigner et pratiquer avec OCaml est le mode Tuareg d'Emacs, qui est le contraire d'un IDE moderne.
De plus, \Jupyter{} note de bas de page{See \url{github.com/louisabraham/domical}} permet à quiconque d'interagir avec des documents entrelaçant du texte, des mathématiques, des chiffres et des cellules de code OCaml, directement depuis un navigateur web et sans installer aucun logiciel, ce qui est un net avantage car OCaml ne s'installe généralement pas par défaut sur les ordinateurs modernes (contrairement au Python qui est livré sur les ordinateurs GNU/Linux et Mac OS X actuellement).


% -------------------------
% \section{Outline of the poster content}
\section*{Aperçu du contenu du poster}

% The poster will present the following points.
Le poster présentera les points suivants.


% \subsection*{Presentation of \Jupyter}
\subsection*{Présentation de \Jupyter}


\begin{itemize}
    \item Qu'est-ce qu'un notebook \Jupyter{} : un environnement de développement intégré (IDE) ``WYSIWYG'' (What-you-see-is-what-you-get) pour (presque) tous les langages de programmation. Par exemple, il peut être utilisé pour des langages dynamiques interprétés, tels que Python \cite{python}, OCaml\footnote{Avec \url{github.com/akabe/ocaml-jupyter}, et d'autres ``noyaux'' pour d'autres langages.}, Julia ou Bash, mais aussi pour des langages compilés, tels que C/C++ etc.

    \item Qu'est-ce que l'écosystème \Jupyter{} : il a commencé sous le nom \texttt{ipython} \cite{ipython} il y a environ 10 ans, conçu pour être utilisé uniquement pour le langage de programmation Python, et de nos jours il a évolué en un écosystème open-source mature.
    Il est utilisé par des centaines de milliers de scientifiques du monde entier, et parmi ses récentes utilisations réussies, on peut noter la toute première image d'un trou noir par Katie Bouman et ses collaborateurs\footnote{Voir par exemple \url{www.nationalgeographic.com/science/2019/04/first-picture-black-hole-revealed}\\\url{-m87-event-horizon-telescope-astrophysics/} et \url{www.bbc.com/news/science-environment-47891902}.}, ou par des lauréats de prix Nobel, comme Paul Romer\footnote{Voir \url{paulromer.net/jupyter-mathematica-and-the-future-of-the-research-paper/}}.
    Les notebooks \Jupyter{} sont une alternative gratuite et open-source à l'EDI propriétaire inclus dans les logiciels MATLAB, Wolfram's Mathematica et MapleSoft's Maple.

    \item Quels problèmes résolvent les notebooks \Jupyter{} ? Pourquoi c'est un outil puissant, à la fois facile à apprendre et à utiliser pour les débutants et puissant pour les utilisateurs experts.
\end{itemize}


\subsection*{Comment installer \Jupyter{} et OCaml et le \emph{kernel} pour OCaml}

En suivant le tutoriel en ligne depuis \url{jupyter.org/install.html}, il est facile d'installer tout l'écosystème \Jupyter{} sur tout ordinateur avec Python et \texttt{pip} ou \texttt{conda} installés.
%
Le logiciel OCaml \cite{ocaml} s'installe aussi facilement, et ensuite l'installation du \emph{kernel} \Jupyter{} spécifique pour utiliser OCaml dans un notebook est simple et rapide \cite{akabe}.


\subsection*{Comment utiliser \Jupyter{} pour écrire des documents simples}

Le poster comprendra une capture d'écran de l'interface utilisateur graphique des notebooks \Jupyter{}.
Il comprendra également des liens vers le tutoriel et la documentation officiels en ligne, qui vous permettront d'apprendre par vous-même.


% \subsection*{Presenting my own usage of \Jupyter{} notebooks}

% I will show how I have been using \Jupyter{} notebooks on a daily basis for my teaching activities, since the last three years.
% I will mainly present examples of resource produced from a \Jupyter{} notebook, and how to convert notebooks to HTML.

% \begin{itemize}
%     \item
%     Using the \textbf{OCaml} language,
%     for the 3rd year (M2) students at ENS de Rennes,
%     I used \Jupyter{} notebooks to write the sheets and solutions of practical sessions used for training our students for the ``modelisation'' oral exam of ``agr{\'e}gation'' nation exam.
%     See \url{nbviewer.jupyter.org/github/Naereen/notebooks/tree/master/agreg/TP_Programmation_2017-18/}

%     \item
%     Using the \textbf{Python} language, for 1st year (L3) students at ENS de Rennes,
%     I wrote a lot of clean and detailed implementations of data structures and algorithms from scratch, for my course on Algorithms in Autumn 2019.
%     See \url{github.com/Naereen/ALGO1-Info1-2019/}

%     \item
%     Using \textbf{Python} and maths,
%     for students in a PSI class (CPGE) at Lyc{\'e}e Joliot-Curie in June 2017 to 2019,
%     I wrote the solutions for practical sessions given as training for the oral ``maths with Python'' exam in Jupyter notebooks, in order to share them easily with the students, and display and work on them during the practical sessions.
%     See \url{perso.crans.org/besson/notebooks/Oraux_CentraleSupelec_PSI__Juin_2019.html}

%     \item
%     With a colleague, we gave a 1 hour tutorial introducing the Julia programming language, in yearly seminar of the IETR lab in June 2018 in Vannes, for about 50 people.
%     We used two \Jupyter{} notebooks during the seminar \url{github.com/pierre-haessig/julia-presentation-ietr2018/},
%     and these slides \url{hal.archives-ouvertes.fr/cel-01830248/}.

% \end{itemize}


% \subsection*{Pointers on how to become a \Jupyter{} expert}
\subsection*{Des conseils pour devenir un expert en \Jupyter}

Le poster contiendra également des liens pour maîtriser l'écosystème \Jupyter{} et les utiliser pour votre propre projet.

\begin{itemize}
    \item Comment utiliser un dépôt GitHub/Bitbucket/GitLab pour héberger des notebooks \Jupyter{}, et les afficher en ligne en utilisant le site \url{nbviewer.jupyter.org/}.
    Voir \url{github.com/Naereen/ALGO1-Info1-2019/} et \url{github.com/Naereen/notebooks/}
    \item Utilisez Binder, Google Colab ou d'autres outils gratuits en ligne, pour ajouter un lien afin que tout utilisateur consultant vos notebooks \Jupyter{} puisse démarrer un environnement interactif, directement depuis son navigateur Web, pour interagir avec le notebook sans rien avoir à installer.
    \item Utilisez les extensions \Jupyter{} pour améliorer l'EDI, par exemple pour ajouter automatiquement une table des matières. \\
    Voir \url{jupyter-contrib-nbextensions.readthedocs.io/en/latest/}.
    \item Suivre les meilleurs exemples, par exemple le célèbre Peter Norvig publie des notebooks très intéressants sur son projet \url{github.com/norvig/pytudes}, depuis 5 ans.
    \item Partagez vos notebooks en ligne, avec vos étudiants et collègues, et recevez leurs commentaires.
\end{itemize}



% ---- Bibliography ----
%
% BibTeX users should specify bibliography style 'splncs04'.
% References will then be sorted and formatted in the correct style.
%
% \bibliographystyle{splncs04}
% \bibliography{mybibliography}
%
\begin{thebibliography}{8}
    \bibitem{jupyter}
    Thomas Kluyver et al.
    \textbf{Jupyter Notebooks - a publishing format for reproducible computational workflows}.
    In F. Loizides and B. Schmidt, editors, Positioning and Power in Academic Publishing: Players, Agents and Agendas, pages 87–90. IOS Press, 2016.

    \bibitem{ocaml}
    INRIA.
    \textbf{OCaml}, version 4.09.0, October 2019. \url{ocaml.org}.

    \bibitem{akabe}
    Akinori Abe et al.
    \textbf{An OCaml kernel for Jupyter (IPython) notebook}.
    Août 2019. \url{github.com/akabe/ocaml-jupyter} et \url{akabe.github.io/ocaml-jupyter}.

    \bibitem{python}
    Python Software Foundation.
    \textbf{Python Language Reference}, version 3.6, October 2017. \url{www.python.org}.

    \bibitem{ipython}
    Fernando Pérez and Brian E. Granger.
    \textbf{IPython: a System for Interactive Scientific Computing}.
    \emph{Computing in Science and Engineering}, 9(3):21–29, May 2007. ISSN 1521-9615. \url{ipython.org}.
\end{thebibliography}
\end{document}
